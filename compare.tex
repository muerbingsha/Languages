\documentclass[12pt, a4pape]{article}

\usepackage[top=1.5cm, bottom=1.5cm, left=1.5cm, right=1.5cm, columnsep=1cm]{geometry}
% table
\usepackage{booktabs}
\usepackage{multirow}
\usepackage{makecell}
% code
\usepackage{listings} % insert code
\usepackage{xcolor}	 % code color
\definecolor{codegreen}{rgb}{0, 0.6, 0}
\definecolor{codegray}{rgb}{0.5,0.5,0.5}
\definecolor{codepurple}{rgb}{0.58,0,0.82}
\definecolor{backcolour}{rgb}{0.95,0.95,0.92}
\lstset{
backgroundcolor=\color{backcolour}, 
commentstyle=\color{codegreen},
keywordstyle=\color{magnenta},
numbers=left, %行号在左侧显示
numberstyle= \small\color{codegray},%行号字体
numbersep=5pt,
rulesepcolor= \color{gray!80}, %代码块边框颜色
breaklines=true,  %代码过长则换行
keywordstyle= \color{blue},%关键字颜色
frame=shadowbox,	%用方框框住代码块
showspaces=false,
showstringspaces=false,
showtabs=false,
tabsize=1
}

\usepackage{graphicx}
\usepackage{tikz}

\usepackage{fontspec} % require LuaLatex engine
\setmainfont{Times New Roman}

\usepackage{hyperref}
\usepackage{float}


%---end preamble---

\begin{document}
\title{\textbf{Comparison between Languages}}
\author{Shark Deng}
\date{March 4 2019}
\maketitle

\section{Introduction}
Development of compilers, such as clang and llvm, booms many computer languages. To name a few, java, swift, php, go, javascript, html, python, c, c++, c\#. In this article, we will go through building blocks of a language. With these elements, we may pick up a new language very quickly or even can create one by llvm.

\section{Overview}
	Languages can be divided into these categories:
	\begin{description}
		\item[C Famility] C, C++, C\#, Objective-C
		\item[Based on C] Java, Javascript, Swift, Php, Python
		\item[ML] XML, HTML, YAML
		\item[SQL(Structured Query Language)]  MySql
		\item[Shell] AppleScript
		\item[Other] Latex, Markdown
	\end{description}
	

	
	\subsection{Oriented}
	Object Oriented Programming(OOP)\footnote{https://www.programiz.com/python-programming/object-oriented-programming}, composed of attributes and behaviors, includes four aspects: encapsulation, inheritance, and polymorphism.
	\begin{table}[H]
	\begin{tabular}{l|l}
	\toprule
	Encapsulation & Hiding the private details of a class from other objects. \\
	\hline
	Inheritance & A process of using details from a new class without modifying existing class.\\
	\hline
	Polymorphism & A concept of using common operation in different ways for different data input. \\
	\bottomrule
	\end{tabular}
	\caption{Source: https://www.programiz.com/python-programming/object-oriented-programming}
	\end{table}
	
	\begin{description}
		\item[Object-oriented] C++, C\#, Objective-C, Java, Javascript, Python,Php
		\item[Procedure-oriented] C
		\item[Protocol-oriented] Swift
		\item[Graphic-oriented] Swift
	\end{description}
	
	\subsection{Strong vs Weak Type}
	Strong type means all data must be typed while weak not.
	\textbf{Swift} is strong type because it's designed to run fast. So most of type check works is on programmers and IDE. \textbf{Php},  however, use Zend engine especially universal variable \emph{zval} to parse variables, so its speed will be mush slower. \textbf{C} combines strong and weak types. In normal programming, it needs designate type but in micro programming, type is missed. Other languages, such as \textbf{Java} is strong type.
	\begin{description}
		\item[Strong Type] Swift, Java, C
		\item[Weak Type] Php, Python, C(micro)
	\end{description}
	
	
	\subsection{Static vs Dynamic}
	It depends on whether the type check is conducted at compile-time or run-time. 
	\begin{description}
		\item[Static]
		\item[Dynamic]
		\item[Static \& Dynamic] Java
	\end{description}
	
	
	\subsection{Generics}
	\begin{description}
		\item[Support] Java, Swift
		\item[Not support]
	\end{description}
		
	\subsection{Compile}
		\subsubsection{Java}
			\textbf{Code on terminal} 
			None \\
			\textbf{Compile file on terminal} 
			\begin{lstlisting}
				$ javac HelloWorld.java
				$ java HelloWorld
			\end{lstlisting}
			\textbf{Code on IDE} IntelliJ

		\subsubsection{Swift}
			\textbf{Code on terminal}  
			\begin{lstlisting}
				$ swift
Welcome to Apple Swift version 4.2.1. Type :help for assistance.
				1> var a = 88
a: Int = 88
				2> let b = 66
b: Int = 66
				3> let c = "Hello!"
c: String = "Hello!"
				4> let d = c + String(b)
d: String = "Hello!66"
				5> import Foundation
			\end{lstlisting}
			Control + d to exit. \\
			\textbf{Compile file on terminal}  \\
			\textbf{Code on IDE} Xcode
		
		\subsubsection{Python}
			\textbf{Code on terminal}  
			\begin{lstlisting}
			$ python
			>>> print("Hello World")
			>>> a = 10
			>>> import tensorflow
			>>> quit()
			\end{lstlisting}
			\textbf{Compile file on terminal}  
			\begin{lstlisting}
			$ touch a.py
			$ echo "print('Hello World')" >  a.py
			$ cat a.py 	
			$ python a.py
			$ rm a.py
			\end{lstlisting}
			\textbf{Code on IDE} Pycharm 
		
		\subsubsection{Objective-C}
			\textbf{Compile file on terminal}
			\begin{lstlisting}
			$ touch source.m
			$ gcc -framework Foundation source.m -o source
			$ ./source
			\end{lstlisting}
			\textbf{Code on IDE} Xcode
			
		\subsubsection{Php}
			\textbf{Code on IDE} PhpStorm
		
	\subsection{Scalability}
		\subsubsection{Java}
		Some languages have rich libraries. \\
		\begin{lstlisting}
		
		\end{lstlisting}
		\textbf{Dependency Management} \\
		Maven.
		
		\subsubsection{Swift}
		\begin{lstlisting}
		import Foundation
		\end{lstlisting}
		\textbf{Dependency Management} \\
		Codpods
		
		\subsubsection{Python}
		\textbf{Dependency Management} \\
		.yml

	\subsection{Compile System}
	
	\subsection{namespace}
	
	
\section{Variable}
	
\section{Primitive Data Type}
	\subsection{Java}
	\begin{table}[H]
	\centering
	\begin{tabular}{c|l|l|c|c}
	\toprule
	No. & Type & Description & Range & Default \\
	\toprule
	1 & byte & 1-byte (8-bit) / signed / 2's complement integer & -128 - 127 & 0 \\
	\hline
	2 & short & 2-byte (16-bit) / signed / 2's complement & -32768 - 32767 & 0 \\
	\hline
	3 & int & 4-byte (32-bit) / signed / 2's complement & $-2^{31} - 2^{31}-1$ & 0 \\
	\hline
	4 & long & 8-byte (64-bit) / signed 2's complement & $-2^{63} - 2^{63}-1$ & 0 \\
	\hline
	5 & float & (32-bit) single precision floating number  & & 0.0f \\
	\hline
	6 & double & (64-bit) double precision floating number & & 0.0d \\
	\hline
	7 & boolean & logically just a single bit &  true, false & false \\
	\hline
	8 & char & 2-byte (16-bit) Unicode character & 0 - 65535  & 0 \\
	\bottomrule
	\end{tabular}
	\end{table}
	
	\subsection{Python3}
	\begin{table}[H]
	\centering
	\begin{tabular}{c|l|l|c|c}
	\toprule
	No. & Type & Description & Range & Default \\
	\hline
	\multirow{4}{*}{1} & \multirow{4}{*}{Number} & int & &  \\
	& & float & & \\
	& & bool &  &\\
	& & complex & & \\
	\hline
	2 & String & Non-changeable & & \\
	\hline
	3 & Tuple & Non-changeable & & \\
	\hline
	4 & List & Changeable & & \\
	\hline
	5 & Set & Changeable & & \\	
	\hline
	6 & Dictionary & Changeable & & \\
	\bottomrule
	\end{tabular}
	\end{table}
	Use the code to check type
	\begin{lstlisting}
	type(10) # <class 'int'>
	type(5.5) # <class 'float'>
	type(True) # <class 'bool'>
	type(4+3i) # <class 'complex'>
	isinstance(10, int) # True
	\end{lstlisting}


	\subsection{Swift}
	
	\subsection{Objective-C}
	\footnote{$https://en.wikipedia.org/wiki/C_data_types$}
	\subsection{C}
	
	
\section{Data Structure}
	\subsection{Common}
	Tuple, array, dict, set, map. \\
	
	In next section, we will discuss specific data structures in these language.
	\subsection{Java}
		\subsection{Array}
		\begin{lstlisting}{ArrayTest.java}
public class ArrayTest {
    public static void main(String[] args) {
        // declare: don't allocate memory
        double[] a;

        // initialize 1: allocate memory
        a = new double[10];

        // give value
        a[0] = 0.0;
        a[1] = 0.1;
        a[2] = 0.2;
        a[3] = 0.3;
        a[4] = 0.4;

        // initialize 2
        double[] b = {0.1, 0.4, 0.6, 0.3};


        // display 1
        for (int i=0; i<a.length; i++){
            System.out.println(a[i]);
        }

        // display 2
        for (double ele: b){
            System.out.println(ele);
        }


        double[] c = a; // same memory address, two references.
        c = new double[7]; 
    }
}
		\end{lstlisting}
		
		
		\subsection{List}
		\subsubsection{ArrayList}
		ArrayList is dynamic array, which means its length can dynamically increase.But it is not thread safe.
		\subsubsection{HashMap}
		\subsubsection{LindedHashMap}
		\subsubsection{TreeMap}
		
		\href{http://www.jobyme88.com}{stMap.java}


\section{Control Flow - 5}	
	\subsection{Selection - 3}
		\subsubsection{If then else}
		\subsubsection{Switch}
		\subsubsection{Try ... except ...}
	\subsection{Iteration - 2}
		\subsubsection{For}
		\subsubsection{While}
		

\section{Function}
	\subsection{Define}
		\subsubsection{Python}
		\begin{lstlisting}[language={python}]
		def main():
   			print('hello')
			
		main()
		\end{lstlisting}
	
	\subsection{Lambda}
	Also called callback.
	
\section{Object \& Class}
	\subsection{Access Level}
		\subsubsection{Swift - 5}
		\begin{table}[H]
		\centering
		\begin{tabular}{|c|c|l|c|}
		\toprule
		name & specifier & access & example \\
		\hline
		open & & outside module (read and modify) & \\
		\hline
		public & & outside module (read) & \\
		\hline
		internal & & inside module & \\
		\hline
		fileprivate & & inside file & \\
		\hline
		private & & inside class & \\
		\bottomrule
		\end{tabular}
		\end{table}
		
		\subsubsection{Java - 4}
		\begin{table}[H]
		\centering
		\begin{tabular}{|c|c|l|c|}
		\toprule
		name & specifier & access & example \\
		\hline
		public & & - & \\
		\hline
		protected & & inside this class and its children & \\
		\hline
		private & & inside this class & \\
		\hline
		default(package private) & & & \\
		\bottomrule
		\end{tabular}
		\end{table}
		
		\subsubsection{Python - 3}
		\begin{table}[H]
		\centering
		\begin{tabular}{|c|c|l|c|}
		\toprule
		name & specifier & access & example \\
		\hline
		public & - & - & self.public = 10  \\
		\hline
		protected & single underline & - & self.\_protected = 10 \\
		\hline
		private & double underlines & inside this class & self.\_\_private = 10\\
		\bottomrule
		\end{tabular}
		\end{table}
		

	\subsection{Define}
		A class has \textbf{constructor}, \textbf{destructor}. These will be detailed in following table of responding languages:
		\subsubsection{Python}
		Example code for Magic Methods is available here \href{https://github.com/muerbingsha/compsumm/blob/master/learn/python/Test5.py}{Test5.py}
		\begin{table}[H]
		\centering
		\begin{tabular}{|c|l|}
		\toprule
		Magic Methods & Meaning \\
		\toprule
		\_\_new\_\_ & create a new instance \\
		\_\_init\_\_ & constructor(initialize a new instance)  \\
		\_\_del\_\_ & desctructor \\
		\hline
		\_\_str\_\_ & print(obj) \\
		\_\_repr\_\_ & obj(on terminal) \\
		\hline
		\_\_getitem\_\_ & \\
		\_\_setitem\_\_ & \\
		\hline
		\_\_cmp\_\_ & \\
		\_\_eq\_\_ & = \\
		\_\_ne\_\_ & != \\
		\_\_lt\_\_ & < \\
		\_\_gt\_\_ & > \\
		\_\_le\_\_ & <= \\
		\_\_ge\_\_ & >= \\
		\hline
		\_\_add\_\_ & + \\
		\_\_sub\_\_ & - \\
		\_\_floordiv\_\_ & // \\
		\_\_truediv\_\_ & / \\
		\_\_mod\_\_ & \% \\
		\_\_pow\_\_ & ** \\
		\_\_lshift\_\_ & << \\
		\_\_rshift\_\_ & >> \\
		\_\_and\_\_  & \& \\
		\_\_xor\_\_ & \\
		\_\_or\_\_ & \\
		\bottomrule
		\end{tabular}
		\caption{Magic methods}
		\end{table}
		
		Python use \textbf{decorator} to realize static class and so on. Pre-made decorators are listed in the following table \ref{tab-decorator} and relevant example is \href{https://github.com/muerbingsha/compsumm/blob/master/learn/python/Test1.py}{Test1.py}. Custom decorator is exampled here \href{https://github.com/muerbingsha/compsumm/blob/master/learn/python/Test6.py}{Test6.py}. 
		\begin{table}[H]
		\centering
		\begin{tabular}{|l|l|c|}
		\toprule
		Method Decorator & Meaning & Example \\
		\toprule
		@staticmethod &  & \\
		\hline
		@classmethod & &  \\
		\hline
		@property & getter and setter & \href{https://github.com/muerbingsha/compsumm/blob/master/learn/python/Test3.py}{Test3.py} \\
		\bottomrule
		\end{tabular}
		\caption{Pre-made decorators}
		\label{tab-decorator}
		\end{table}
		
		
	\subsection{Inheritance}
		\subsubsection{java}
	
		
	\subsection{Interface}
		\subsubsection{Java}
		has 
		\subsubsection{C}
		Because C is procedure-oriented. It doesn't have interface.
		\subsubsection{C++}
		\subsubsection{C\#}
		\subsubsection{Objective-C}
		\subsubsection{Swift}
		\subsubsection{Php}
		
	
	\begin{figure}
		\begin{tikzpicture}
		\end{tikzpicture}
	\end{figure}
	
	\subsection{Annotation}
	
	

\section{Characteristics}
	\subsection{Objective-C}
		\subsubsection{Category}
		\subsubsection{Extension}
		\subsubsection{Protocol}


\section{Enum}
\section{Singleton}
\section{JavaFXApplication}


	
\section{Code Standards}

\clearpage
\lstlistoflistings

\end{document}